\documentclass{beamer}
\usetheme[progressbar=frametitle]{metropolis}
\setbeamertemplate{frame numbering}[fraction]
\useoutertheme{metropolis}
\useinnertheme{metropolis}
\usefonttheme{metropolis}

% Barra de progreso más ancha
\makeatletter
\setlength{\metropolis@titleseparator@linewidth}{2pt}
\setlength{\metropolis@progressonsectionpage@linewidth}{2pt}
\setlength{\metropolis@progressinheadfoot@linewidth}{2pt}
\makeatother

\usepackage[utf8]{inputenc}
\usepackage[english]{babel}
\usepackage[compatibility=false,justification=centering]{caption}
\usepackage[listings,theorems]{tcolorbox}
\usepackage{minted}
\usepackage{url}
\usepackage{svg}

\usemintedstyle{friendly}

\DeclareCaptionFont{captionfont}{\fontsize{8}{8}\selectfont}
\captionsetup{font=captionfont}

\title{Especificación y Verificación Formal de Sistemas Distribuidos con TLA+}
\subtitle{02. Principios de TLA+ y PlusCal}
\author{Axel Suárez Polo}
\institute{BUAP}
\date{\today}

\begin{document}

\begin{frame}
  \titlepage
\end{frame}

\begin{frame}[t]
  \frametitle{Contenidos}
  \tableofcontents
\end{frame}

\section{Modelando programas con TLA+}

\begin{frame}
  \frametitle{Modelando programas con TLA+}

  \begin{itemize}
    \item La especificación de un sistema puede ir desde simple prosa, hasta
          una especificación matemática, como ocurre en la ciencia y la ingeniería.
          \begin{itemize}
            \item En la realidad, los planetas tienen montañas, océanos, olas, clima, etc
            \item Pero los podemos modelar como puntos de masa con posición y momento
          \end{itemize}
    \item TLA+ permite hacer lo mismo pero con los programas
  \end{itemize}
\end{frame}

\section{El modelo de comportamientos}

\begin{frame}
  \frametitle{El modelo de comportamientos}

  \begin{itemize}
    \item TLA+ utiliza el \emph{modelo de comportamientos}
    \item La ejecución de un programa es representada por un \textbf{comportamiento}.
    \item Un \textbf{comportamiento} es una secuencia ordenada de \textbf{estados}, ya sea finita o infinita.
    \item Un \textbf{estado} es una asignación de valores a variables.

    \item Un \textbf{programa} es modelado por un conjunto de \textbf{comportamientos}.
  \end{itemize}
\end{frame}

\begin{frame}
  \frametitle{El modelo de comportamientos}

  \begin{figure}[h]
      \centering
      \includesvg[width=\textwidth]{assets/semaforo_comportamiento}
      \caption{El comportamiento de un semáforo}
      \label{fig:sem}
  \end{figure}

\end{frame}

\begin{frame}
  \frametitle{El modelo de comportamientos}

  \begin{itemize}
    \item \emph{Un programa es modelado por todas sus posibles ejecuciones.}
    \item TLA+ nos permite especificar de forma precisa y concisa \emph{todas} las ejecuciones posibles de un programa.
  \end{itemize}
\end{frame}

\section{Un semáforo en TLA+}

\begin{frame}[fragile]
  \frametitle{Semáforo en TLA+}

  \begin{listing}[H]
    \begin{center}
      \begin{minipage}{0.7\textwidth}
        \begin{minted}[autogobble,linenos,mathescape,fontsize=\scriptsize]{idris}
            VARIABLE color
            SInit == color = "verde"
            SNext == IF color = "verde"
                     THEN color' = "amarillo"
                     ELSE IF color = "amarillo"
                          THEN color' = "rojo"
                          ELSE color' = "verde"
            Spec == SInit /\ [][SNext]_color

            TypeInvariant == color \in {"verde", "amarillo", "rojo"}
            ----
            THEOREM Spec => []TypeInvariant
        \end{minted}
      \end{minipage}
    \end{center}
    \caption{Semáforo en TLA+}
    \label{lst:sem_tla}
  \end{listing}
\end{frame}

\begin{frame}[fragile]
  \frametitle{Sintaxis}

  La sintaxis es la siguiente:

  \begin{itemize}
    \item \mintinline{idris}|VARIABLE v1, v2, ...| declara las variables $v1, v2, ...$
    \item \mintinline{idris}|Ident == e1| define el identificador $Ident$ como la expresión $e1$.
          Es idéntico a \mintinline{c}|#define Ident e1| en C.
          \footnote{Permite la \emph{Transparencia Referencial}.}
    \item \mintinline{idris}|e1 = e2| realiza la aserción de que $e1$ es \emph{igual} a $e2$.
          Parecido a \mintinline{c}|e1 == e2| en C,
          pero también puede especificar el valor inicial de una variable.
  \end{itemize}
\end{frame}

\begin{frame}[fragile]
  \frametitle{Sintaxis}

  La sintaxis es la siguiente:

  \begin{itemize}
    \item \mintinline{idris}|id' = e1| asigna al \textbf{siguiente estado} de la variable $id$
          el valor de la expresión $e1$.
    \item \mintinline{idris}|e1 /\ e2| \emph{and} lógico. Parecido a \mintinline{c}|e1 && e2| en C
          \footnote{La diferencia es que \mintinline{idris}|e1 /\ e2| no infiere control de flujo.}.
    \item \mintinline{idris}|[][e1]_e2| aserción de que la propiedad $e1$ se cumple \textbf{siempre}
          o no ocurre cambio en las variables.
          \footnote{Considerar que no haya cambio en las variables permite el \emph{stuttering}.}
  \end{itemize}
\end{frame}

\begin{frame}[fragile]
  \frametitle{Sintaxis}

  La sintaxis es la siguiente:

  \begin{itemize}
    \item \mintinline{idris}|e1 \in e2| realiza la aserción de que $e1$
          pertenece al conjunto $e2$. Si $e1$ es un identificador y es el
          estado inicial, entonces $e1$ toma todos los valores de $e2$.
    \item \mintinline{idris}|{e1, e2, ...}| el conjunto conformado por $e1$, $e2$, etc.
    \item \mintinline{idris}|e1 => e2| $e1$ implica lógicamente $e2$.
  \end{itemize}
\end{frame}

\begin{frame}[fragile]
  \frametitle{Transparencia Referencial}

  \emph{Una variable puede ser reemplazada por su valor sin alterar el programa.}

  No se cumple en lenguajes imperativos en general, pero sí en TLA+ y en lenguajes funcionales puros.

  \begin{listing}[H]
    \begin{center}
      \begin{minipage}{0.7\textwidth}
        \begin{minted}[autogobble,linenos,mathescape,fontsize=\tiny]{c}
        #include <stdio.h>

        int f() {
          printf("Hola\n");
          return 2;
        }

        int main() {
          int x = f();
          int valor = x + x;
          // int valor = f() + f();
          printf("%d\n", valor);
        }
        \end{minted}
      \end{minipage}
    \end{center}
    \caption{Transparencia Referencial no aplica en C}
    \label{lst:c_ref}
  \end{listing}
\end{frame}

\begin{frame}[fragile]
  \frametitle{Stuttering}

  Permitir pasos que no cambian las variables (\emph{stuttering steps}) permite
  una mejor \textbf{composición} (reutilización) de las especificaciones.

  \begin{figure}[h]
      \centering
      \includesvg[height=1.6in]{assets/stuttering}
      \caption{Composición de especificaciones con \emph{stuttering}.}
      \label{fig:sem}
  \end{figure}
\end{frame}

\section{Un semáforo en PlusCal}

\begin{frame}[fragile]
  \frametitle{Semáforo en PlusCal}

  \begin{listing}[H]
    \begin{center}
      \begin{minipage}{0.5\textwidth}
        \begin{minted}[autogobble,linenos,mathescape,fontsize=\scriptsize]{idris}
            variable color = "verde";

            while TRUE do
              if color = "verde" then
                color := "amarillo";
              elsif color = "amarillo" then
                color := "rojo";
              else
                color := "verde";
              end if;
            end while;
        \end{minted}
      \end{minipage}
    \end{center}
    \caption{Semáforo en PlusCal}
    \label{lst:sem_pluscal}
  \end{listing}
\end{frame}

\end{document}
