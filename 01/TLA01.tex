\documentclass{beamer}
\usetheme[progressbar=frametitle]{metropolis}
\setbeamertemplate{frame numbering}[fraction]
\useoutertheme{metropolis}
\useinnertheme{metropolis}
\usefonttheme{metropolis}

% Barra de progreso más ancha
\makeatletter
\setlength{\metropolis@titleseparator@linewidth}{2pt}
\setlength{\metropolis@progressonsectionpage@linewidth}{2pt}
\setlength{\metropolis@progressinheadfoot@linewidth}{2pt}
\makeatother

\usepackage[utf8]{inputenc}
\usepackage[english]{babel}
\usepackage[compatibility=false]{caption}
\usepackage[listings,theorems]{tcolorbox}
\usepackage{minted}
\usepackage{url}

\usemintedstyle{friendly}

\DeclareCaptionFont{captionfont}{\fontsize{8}{8}\selectfont}
\captionsetup{font=captionfont}

\title{Especificación y Verificación Formal de Sistemas Distribuidos con TLA+}
\subtitle{01. Introducción}
\author{Axel Suárez Polo}
\institute{BUAP}
\date{\today}

\begin{document}

\begin{frame}
  \titlepage
\end{frame}

\begin{frame}[t]
  \frametitle{Contenidos}
  \tableofcontents
\end{frame}

\section{¿Por qué aprender TLA+?}

\begin{frame}
  \frametitle{¿Por qué aprender TLA+?}

  \begin{itemize}
    \item Escribir la \emph{especificación} de un sistema ayuda a entenderlo
    \item Es buena idea entender un sistema antes de implementarlo.
    \item Por lo tanto, es buena idea \emph{especificar} un sistema antes de implementarlo.
  \end{itemize}
\end{frame}

\begin{frame}
  \frametitle{¿Por qué aprender TLA+?}

  \begin{itemize}
    \item La especificación de un sistema puede ir desde simple prosa, hasta
          una especificación matemática, como ocurre en la ciencia y la ingeniería.
          \begin{itemize}
            \item En la realidad, los planetas tienen montañas, océanos, olas, clima, etc
            \item Pero los podemos modelar como puntos de masa con posición y momento
          \end{itemize}
  \end{itemize}
\end{frame}

\begin{frame}
  \frametitle{¿Por qué aprender TLA+?}

  \begin{itemize}
    \item Este mismo método científico lo podemos aplicar a los programas:
          \begin{itemize}
            \item En la realidad, hay procesos físicos ocurriendo en las computadoras,
              transistores cambiando de estado, diferencias de voltaje, etc. \\
              Incluso a un nivel de abstracción más alto existen direcciones de
              memoria, estados de registros, etc.
            \item Pero podemos describir lo que hace una computadora con modelos como las
              \textbf{máquinas de Turing}, el \textbf{cálculo lambda}, etc.
          \end{itemize}
  \end{itemize}
\end{frame}

\begin{frame}
  \frametitle{¿Por qué aprender TLA+?}

  \begin{itemize}
    \item TLA+ nos da un modelo para razonar sobre casi cualquier sistema discreto
          \cite{lamport1999specifying}.
          \begin{itemize}
            \item Diseño de hardware \cite{lamport2001wildfire}.
            \item Sistemas concurrentes y distribuidos \cite{newcombe2015amazon}.
            \item Sistemas operativos \cite{verhulst2011formal}.
            \item Algoritmos como 2PC, Paxos, alojamiento de memoria, etc \cite{githubtla}.
          \end{itemize}
  \end{itemize}
\end{frame}

\section{¿Qué es TLA+ y TLC?}

\begin{frame}
  \frametitle{¿Qué es TLA+?}

  \begin{itemize}
    \item TLA+ un lenguaje de alto nivel para modelar sistemas digitales. \cite{tlacourse}.
          \begin{itemize}
            \item \textbf{sistemas digitales} abarca tanto algoritmos como sistemas de computadoras.
            \item \textbf{alto nivel} hace referencia a que se está hablando a nivel de diseño y no de código ejecutable.
          \end{itemize}
  \end{itemize}
\end{frame}

\begin{frame}
  \frametitle{¿Qué es TLC?}

  \begin{itemize}
    \item TLC es una herramienta para verificar estos modelos automáticamente.
          \begin{itemize}
            \item Estas herramientas permiten encontrar y corregir errores de diseño. Hoy en día los errores de diseño ocupan los primeros lugares en vulnerabilidades de seguridad \cite{owasp4}.
          \end{itemize}
  \end{itemize}
\end{frame}

\section{¿Por qué no aprender TLA+?}

\begin{frame}
  \frametitle{¿Por qué no aprender TLA+?}

  \begin{itemize}
    \item En el espectro de los métodos formales, TLA+ presenta una dificultad mayor a la de los sitemas de tipos de lenguajes como C o Java \cite{fisher2017hacms}.
    \item Existe software simple de escribir y software difícil de escribir.
    \item Existe software crítico y software que no lo es.
    \item \textbf{No se puede verificar que el software implementado efectivamente sigue la especificación.}
  \end{itemize}
\end{frame}

\begin{frame}[allowframebreaks]
        \frametitle{Referencias}
        \bibliographystyle{amsalpha}
        \bibliography{./refs.bib}
\end{frame}

\end{document}
