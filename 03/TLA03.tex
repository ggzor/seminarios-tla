\documentclass{beamer}
\usetheme[progressbar=frametitle]{metropolis}
\setbeamertemplate{frame numbering}[fraction]
\useoutertheme{metropolis}
\useinnertheme{metropolis}
\usefonttheme{metropolis}

% Barra de progreso más ancha
\makeatletter
\setlength{\metropolis@titleseparator@linewidth}{2pt}
\setlength{\metropolis@progressonsectionpage@linewidth}{2pt}
\setlength{\metropolis@progressinheadfoot@linewidth}{2pt}
\makeatother

\usepackage[utf8]{inputenc}
\usepackage[english]{babel}
\usepackage[compatibility=false,justification=centering]{caption}
\usepackage[listings,theorems]{tcolorbox}
\usepackage{minted}
\usepackage{url}
\usepackage{svg}
\usepackage{amsmath}
\usepackage{mathtools}
\usepackage{graphicx}

\usemintedstyle{friendly}

\DeclareCaptionFont{captionfont}{\fontsize{8}{8}\selectfont}
\captionsetup{font=captionfont}

\title{Especificación y Verificación Formal de Sistemas Distribuidos con TLA+}
\subtitle{03. Máquinas de estados}
\author{Axel Suárez Polo}
\institute{BUAP}
\date{\today}

\begin{document}

\begin{frame}
  \titlepage
\end{frame}

\begin{frame}[t]
  \frametitle{Contenidos}
  \tableofcontents
\end{frame}

\section{Modelando un programa de C}

\begin{frame}[fragile]
  \frametitle{Programa en C}

  ¿Cómo modelar el siguiente programa?

  \begin{listing}[H]
    \begin{center}
      \begin{minipage}{0.7\textwidth}
        \inputminted{c}{code/program.c}
      \end{minipage}
    \end{center}
    \caption{Programa en C}
    \label{lst:cprogram_1}
  \end{listing}
\end{frame}

\begin{frame}
  \frametitle{El modelo de comportamientos}

  TLA+ utiliza el \emph{modelo de comportamientos}.

  \begin{itemize}
    \item La ejecución de un programa es representada por un \textbf{comportamiento}.
    \item Un \textbf{comportamiento} es una secuencia ordenada de \textbf{estados}, ya sea finita o infinita.
    \item Un \textbf{estado} es una asignación de valores a variables.

    \item Un \textbf{programa} es modelado por un conjunto de \textbf{comportamientos}.
  \end{itemize}
\end{frame}

\begin{frame}
  \frametitle{El modelo de comportamientos}

  \begin{figure}[h]
      \centering
      \includesvg[width=\textwidth]{assets/semaforo_comportamiento}
      \caption{El comportamiento de un semáforo}
      \label{fig:sem_behavior}
  \end{figure}

\end{frame}

\begin{frame}[fragile]
  \frametitle{Pasos para modelar un programa}

  Para modelar un programa o sistema con TLA+ tenemos que saber 3 cosas:

  \begin{itemize}
    \item \textbf{Variables} del sistema
    \item \textbf{Estado} inicial
    \item Relación entre el \textbf{estado} actual y el \textbf{estado} siguiente
  \end{itemize}
\end{frame}

\begin{frame}[fragile]
  \frametitle{Programa en C}

  ¿Cuáles son las \textbf{variables},
                  \textbf{estado inicial} y
                  \textbf{relación entre estados} del programa?

  \begin{listing}[H]
    \begin{center}
      \begin{minipage}{0.7\textwidth}
        \inputminted{c}{code/program.c}
      \end{minipage}
    \end{center}
    \caption{Programa en C}
    \label{lst:cprogram_2}
  \end{listing}
\end{frame}

\begin{frame}[fragile]
  \frametitle{Pensando en comportamientos}

  Una forma de determinar esto es pensando en los comportamientos del programa.

  \begin{figure}
    \begin{center}
      \begin{equation*}
        \begingroup
        \setlength\arraycolsep{1pt}
        \begin{aligned}
          \begin{bmatrix*}
            \textbf{i} &:& 0
          \end{bmatrix*}
          \rightarrow
          \begin{bmatrix*}
            \textbf{i} &:& 10
          \end{bmatrix*}
          \rightarrow
          \begin{bmatrix*}
            \textbf{i} &:& 11
          \end{bmatrix*}
        \end{aligned}
        \endgroup
      \end{equation*}
      \pause
      \begin{equation*}
        \begingroup
        \setlength\arraycolsep{1pt}
        \begin{aligned}
          \begin{bmatrix*}
            \textbf{i} &:& 0
          \end{bmatrix*}
          \rightarrow
          \begin{bmatrix*}
            \textbf{i} &:& 99
          \end{bmatrix*}
          \rightarrow
          \begin{bmatrix*}
            \textbf{i} &:& 100
          \end{bmatrix*}
        \end{aligned}
        \endgroup
      \end{equation*}
    \end{center}

    \caption{Posibles comportamientos del programa}
    \label{lst:cprogram_behavior}
  \end{figure}

\end{frame}

\section{Programar 1: Programa en C}

\begin{frame}
  \frametitle{El modelo de comportamientos}

  \begin{itemize}
    \item Un \emph{comportamiento} \textbf{siempre es lineal}.
    \item El modelo de comportamientos requiere pensar sobre las relaciones entre los
          estados.
    \item El \textbf{conjunto de comportamientos} es la especificación del programa.
  \end{itemize}
\end{frame}

\begin{frame}[fragile]
  \frametitle{Pensando en comportamientos}

  Tenemos que usar una fórmula lógica para representar la relación entre
  estados, lo que es imposible con las variables actuales.

  \begin{figure}
    \begin{center}
      \begin{equation*}
        \begingroup
        \setlength\arraycolsep{1pt}
        \begin{aligned}
          \begin{bmatrix*}
            \textbf{i} &:& 0
          \end{bmatrix*}
          \rightarrow
          \begin{bmatrix*}
            \textbf{i} &:& 10
          \end{bmatrix*}
          \rightarrow
          \begin{bmatrix*}
            \textbf{i} &:& 11
          \end{bmatrix*}
        \end{aligned}
        \endgroup
      \end{equation*}
      \begin{equation*}
        \begingroup
        \setlength\arraycolsep{1pt}
        \begin{aligned}
          \begin{bmatrix*}
            \textbf{i} &:& 0
          \end{bmatrix*}
          \rightarrow
          \begin{bmatrix*}
            \textbf{i} &:& 11
          \end{bmatrix*}
          \rightarrow
          \begin{bmatrix*}
            \textbf{i} &:& 12
          \end{bmatrix*}
        \end{aligned}
        \endgroup
      \end{equation*}
    \end{center}

    \caption{Posibles comportamientos del programa}
    \label{lst:cprogram_behavior}
  \end{figure}

\end{frame}

\begin{frame}[fragile]
  \frametitle{Estado oculto}

  Los lenguajes de programación convencionales ocultan estado en diversas
  formas: \textbf{variables},
          \textbf{apuntador de instrucción},
          \textbf{pila de llamadas},
          \textbf{memoria dinámica}, etc.

  \begin{listing}[H]
    \begin{center}
      \begin{minipage}{0.7\textwidth}
        \begin{minted}[autogobble,linenos,mathescape,fontsize=\scriptsize]{c}
          int i = 0;             // pc = "start"
          int main() {
            i = obtenerNumero(); // pc = "middle"
            i = i + 1;           // pc = "end"
          }
        \end{minted}
      \end{minipage}
    \end{center}
    \caption{El programa en C con anotaciones}
    \label{lst:cprogram_hidden}
  \end{listing}
\end{frame}

\begin{frame}[fragile]
  \frametitle{Pensando en comportamientos}

  TLA+ requiere que pensemos en todo el estado de un sistema para la especificación.

  \begin{figure}
    \begin{center}
      \begin{equation*}
        \begingroup
        \setlength\arraycolsep{1pt}
        \begin{aligned}
          \begin{bmatrix*}
            \textbf{i} &:& 0 \\
            \textbf{pc}&:& ``start"
          \end{bmatrix*}
          \rightarrow
          \begin{bmatrix*}
            \textbf{i} &:& 10 \\
            \textbf{pc}&:& ``middle"
          \end{bmatrix*}
          \rightarrow
          \begin{bmatrix*}
            \textbf{i} &:& 11 \\
            \textbf{pc}&:& ``end"
          \end{bmatrix*}
        \end{aligned}
        \endgroup
      \end{equation*}
      \begin{equation*}
        \begingroup
        \setlength\arraycolsep{1pt}
        \begin{aligned}
          \begin{bmatrix*}
            \textbf{i} &:& 0 \\
            \textbf{pc}&:& ``start"
          \end{bmatrix*}
          \rightarrow
          \begin{bmatrix*}
            \textbf{i} &:& 11 \\
            \textbf{pc}&:& ``middle"
          \end{bmatrix*}
          \rightarrow
          \begin{bmatrix*}
            \textbf{i} &:& 12 \\
            \textbf{pc}&:& ``end"
          \end{bmatrix*}
        \end{aligned}
        \endgroup
      \end{equation*}
    \end{center}

    \caption{Posibles comportamientos del programa}
    \label{lst:cprogram_behavior_good}
  \end{figure}

\end{frame}

\section{Programar 2: Programa en C (revisado)}

\end{document}
